\documentclass[letterpaper]{book}
\usepackage[times,hyper]{Rd}
\usepackage{makeidx}
\usepackage[utf8,latin1]{inputenc}
\makeindex{}
\begin{document}
\chapter*{}
\begin{center}
{\textbf{\huge \R{} documentation}} \par\bigskip{{\Large of \file{AlleleSetIllumina-class.Rd}}}
\par\bigskip{\large \today}
\end{center}
\inputencoding{utf8}
\HeaderA{AlleleSetIllumina-class}{Class to Contain Objects Describing High-Throughput Illumina BeadArrays}{AlleleSetIllumina.Rdash.class}
\aliasA{initialize,AlleleSetIllumina-method}{AlleleSetIllumina-class}{initialize,AlleleSetIllumina.Rdash.method}
\keyword{classes}{AlleleSetIllumina-class}
%
\begin{Description}\relax
Container for high-throughput assays and experimental
metadata. \code{"AlleleSetIllumina"} class is derived from
\code{"\LinkA{eSet}{eSet}"} and requires matrices
\code{intensity}, \code{theta}, and \code{SE} as assay data
members. \code{"AssayData"} objects have the required storage mode
\code{"list"}.
\end{Description}
%
\begin{Section}{Objects from the Class}

\code{new("AlleleSetIllumina", ...\{\})}

\code{new("AlleleSetIllumina",
    phenoData = [AnnotatedDataFrame],
    featureData = [AnnotatedDataFrame],
    experimentData = [MIAME],
    annotation = [character],
    intensity = [matrix],
    theta = [matrix],
    SE = [matrix],
    ...\{\})
  }

\dots,\dots,\ldots,\ldots

Most arguments are optional, however usually the arguments to \code{new}
include at least \code{intensity}, \code{theta}, and \code{SE}. Note
that the \code{phenoData} columns \code{noiseIntensity},
\code{pedigreeID}, and some others are required by several methods.

\code{AlleleSetIllumina} instances may also be created from
\code{\LinkA{BeadSetIlumina}{BeadSetIlumina}} instances using
\code{\LinkA{createAlleleSet}{createAlleleSet}} or from text-files using
\code{\LinkA{createAlleleSetFromFiles}{createAlleleSetFromFiles}}.
\end{Section}
%
\begin{Section}{Slots}
Inherited from \code{"\LinkA{eSet}{eSet}"}:
\begin{description}

\item[\code{assayData}:] Object of class \code{"AssayData"}. List
containing data-matrices with rows corresponding to markers and
columns to samples/arrays. Required members are \code{intensity},
\code{theta}, and \code{SE}. May also include intensities \code{A}
and \code{B}, \code{call}, \code{ped.check.parents}, or any other
suitable matrix.
\item[\code{phenoData}:] Object of class
\code{"AnnotatedDataFrame"} with additional information about the
samples.
\item[\code{featureData}:] Object of class
\code{"AnnotatedDataFrame"} with additional information about the markers.
\item[\code{experimentData}:] Object of class \code{"MIAME"}
\item[\code{annotation}:] Object of class \code{"character"}
\item[\code{.\_\_classVersion\_\_}:] Object of class \code{"Versions"}

\end{description}

\end{Section}
%
\begin{Section}{Extends}
Class \code{"\LinkA{eSet}{eSet.Rdash.class}"}, directly.
Class \code{"\LinkA{VersionedBiobase}{VersionedBiobase.Rdash.class}"}, by class "eSet", distance 2.
Class \code{"\LinkA{Versioned}{Versioned.Rdash.class}"}, by class "eSet", distance 3.
\end{Section}
%
\begin{Section}{Methods}

Class-specific methods:
\begin{description}

\item[initialize] \code{signature(.Object = "AlleleSetIllumina")}

\end{description}


Derived from \code{"\LinkA{eSet}{eSet}"}:
\begin{description}

\item[\code{sampleNames(object)},
\code{sampleNames(object) <- value}:] See \code{"\LinkA{eSet}{eSet}"}
\item[\code{featureNames(object)},
\code{featureNames(object) <- value}:] See \code{"\LinkA{eSet}{eSet}"}
\item[\code{dims(object)}:] See \code{"\LinkA{eSet}{eSet}"}
\item[\code{phenoData(object)},
\code{phenoData(object) <- value}:] See \code{"\LinkA{eSet}{eSet}"}
\item[\code{pData(object)}, \code{pData(object) <- value}:] See \code{"\LinkA{eSet}{eSet}"}
\item[\code{varMetadata(object)}] See \code{"\LinkA{eSet}{eSet}"}
\item[\code{varLabels(object)}] See \code{"\LinkA{eSet}{eSet}"}
\item[\code{featureData(object)},
\code{featureData(object) <- value}:] See \code{"\LinkA{eSet}{eSet}"}
\item[\code{fData(object)}, \code{fData(object) <- value}:] See \code{"\LinkA{eSet}{eSet}"}
\item[\code{fvarMetadata(object)}] See \code{"\LinkA{eSet}{eSet}"}
\item[\code{fvarLabels(object)}] See \code{"\LinkA{eSet}{eSet}"}
\item[\code{assayData(object), assayData(object) <- value}:] See \code{"\LinkA{eSet}{eSet}"}
\item[\code{experimentData(object)}, \code{experimentData(object) <- value}:] See \code{"\LinkA{eSet}{eSet}"}






\item[\code{combine(object,object)}:] See \code{"\LinkA{eSet}{eSet}"}
\item[\code{storageMode(object)}:] See
\code{"\LinkA{eSet}{eSet}"}. Do not assign new values to
\code{storageMode}, as this would render the \code{AlleleSetIllumina}
object invalid.


\end{description}

\begin{description}

\item[\code{initialize(object)}:] Object instantiation, can be called by
derived classes but not usually by the user.
\item[\code{validObject(object)}:] Validity-checking method,
ensuring (1) all assayData components have the same number of
features and samples; (2) the number and names of
\code{phenoData} rows match the number and names of
\code{assayData} columns
\item[\code{as(eSet, "ExpressionSet")}] Convert instance of class \code{"eSet"} to instance of \code{\LinkA{ExpressionSet-class}{ExpressionSet.Rdash.class}}, if possible.
\item[\code{as(eSet, "MultiSet")}] Convert instance of class \code{"eSet"} to instance of \code{\LinkA{MultiSet-class}{MultiSet.Rdash.class}}, if possible.
\item[\code{updateObject(object, ..., verbose=FALSE)}] Update instance to current version, if necessary. Usually called through class inheritance rather than directly by the user. See \code{\LinkA{updateObject}{updateObject}}
\item[\code{updateObjectTo(object, template, ..., verbose=FALSE)}] Update instance to current version by updating slots in \code{template}, if necessary. Usually call by class inheritance, rather than directly by the user. See \code{\LinkA{updateObjectTo}{updateObjectTo}}
\item[\code{isCurrent(object)}] Determine whether version of object is current. See \code{\LinkA{isCurrent}{isCurrent}}
\item[\code{isVersioned(object)}] Determine whether object contains a "version" string describing its structure . See \code{\LinkA{isVersioned}{isVersioned}}
\item[\code{show(object)}] Informatively display object contents.
\item[\code{dim(object)}, \code{ncol}] Access the common
dimensions (\code{dim}) or column numbers (\code{ncol}), of all
memebers (\code{dims}) of \code{assayData}.
\item[\code{object[(index)}:] Conducts subsetting of matrices and
phenoData components 
\item[\code{object\$name}, \code{object\$name<-value}] Access and set \code{name} column in \code{phenoData}
\item[\code{object[[i, ...]]}, \code{object[[i, ...]]<-value}] Access and set column \code{i} (character or
numeric index) in \code{phenoData}. The ... argument can include
named variables (especially \code{labelDescription}) to be added
to varMetadata.

\end{description}

Additional functions:
\begin{description}

\item[assayDataElement(object, element)] Return matrix
\code{element} from \code{assayData} slot of \code{object}.
\item[assayDataElement(object, element) <- value)] Set element
\code{element} in \code{assayData} slot of \code{object} to matrix \code{value}
\item[assayDataElementReplace(object, element, value)] Set element
\code{element} in \code{assayData} slot of \code{object} to matrix \code{value}
\item[assayDataElementNames(object)] Return element names in
\code{assayData} slot of \code{object}
\item[\code{updateOldESet}] Update versions of \code{eSet}
constructued using \code{listOrEnv} as \code{assayData} slot
(before May, 2006).

\end{description}


Standard generic methods:
\begin{description}

\item[\code{initialize(AlleleSetIllumina)}:] Object instantiation, used
by \code{new}; not to be called directly by the user.
\item[\code{validObject(AlleleSetIllumina)}:] Checks the validity of
\code{alleleSetIllumina}.
\item[\code{show(AlleleSetIllumina)}] See \code{"\LinkA{eSet}{eSet}"}
\item[\code{dim(AlleleSetIllumina)}, \code{ncol}] See \code{"\LinkA{eSet}{eSet}"}
\item[\code{AlleleSetIllumina[(index)}:] See \code{"\LinkA{eSet}{eSet}"}
\item[\code{AlleleSetIllumina\$}, \code{AlleleSetIllumina\$<-}] See \code{"\LinkA{eSet}{eSet}"}

\end{description}

\end{Section}
%
\begin{Section}{Warning }
....
\end{Section}
%
\begin{Note}\relax
 \textasciitilde{}\textasciitilde{}further notes\textasciitilde{}\textasciitilde{} 
\end{Note}
%
\begin{Author}\relax
Lars Gidskehaug
\end{Author}
%
\begin{References}\relax
 \textasciitilde{}put references to the literature/web site here \textasciitilde{} 
\end{References}
%
\begin{SeeAlso}\relax
\textasciitilde{}\textasciitilde{}objects to See Also as \code{\LinkA{\textbackslash{}textasciitilde{}\textbackslash{}textasciitilde{}fun\textbackslash{}textasciitilde{}\textbackslash{}textasciitilde{}}{.Rtilde..Rtilde.fun.Rtilde..Rtilde.}}, \textasciitilde{}\textasciitilde{}\textasciitilde{}
or \code{\LinkA{CLASSNAME}{CLASSNAME.Rdash.class}} for links to other classes
\end{SeeAlso}
%
\begin{Examples}
\begin{ExampleCode}
showClass("AlleleSetIllumina")
\end{ExampleCode}
\end{Examples}
\printindex{}
\end{document}
